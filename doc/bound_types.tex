\documentclass[10pt,russian]{article}
\usepackage[T2A]{fontenc}
\usepackage[utf8]{inputenc}
\usepackage[a4paper]{geometry}
\geometry{verbose,tmargin=2cm,bmargin=2cm,lmargin=2cm,rmargin=2cm,headheight=1cm,headsep=1cm,footskip=1cm}
\usepackage{textcomp}
\pagestyle{empty}
\makeatletter

\usepackage{amsthm}
\usepackage{amsfonts}
\usepackage{amssymb}
\usepackage{amsmath}

\renewcommand{\theenumii}{\arabic{enumi}.\arabic{enumii}.}
\renewcommand{\labelenumii}{\theenumii}
\newcommand{\subtype}{\preccurlyeq}
\newcommand{\meet}{\wedge}
\newcommand{\join}{\vee}
\newcommand{\bigmeet}{\bigwedge}
\newcommand{\bigjoin}{\bigvee}

\makeatother

\usepackage{babel}

\begin{document}

\title{Отложенное вычисление типов экстремумов}
\maketitle

\section{Использование свойств решётки для вывода типов}

Допустим, имеется набор формул, включающий в себя:

\begin{equation}
\label{upper_bounds}
\begin{split}
A &\subtype cons(\alpha_{1,1}, \dotsc, \alpha_{1,m}) \\
  &... \\
A &\subtype cons(\alpha_{n,1}, \dotsc, \alpha_{n,m}) \text{, где}
\end{split}
\end{equation}
$A$ -- типовая переменная; $\alpha_{i,j}$ -- произвольные типы, хотя бы один из
которых является типовой переменной\footnote{она же fresh type}; $cons$ -- некий
типовой конструктор (структура, список, предикат и т.п.) Каждая из этих формул
определяет верхнюю грань для $A$. Также допустим, что других ограничений сверху
у $A$ нет.


Тогда, т.к. типы образуют решётку по операции $\subtype$, должно выполняться

\begin{equation}
\label{bigmeet}
A \subtype {\displaystyle\bigmeet_{i=1}^n cons(\alpha_{i,1}, \dotsc, \alpha_{i,m})}
\end{equation}

В случае, когда в \eqref{upper_bounds} отношения направлены в другую сторону,
в формуле выше используется супремум.

Если инфинум в правой части \eqref{bigmeet} отличен от $\bot$, и если его можно
вычислить, получаем

\begin{equation*}
A \subtype cons(\beta_1, \dotsc, \beta_m) \text{, где}
\end{equation*}

$\beta_i$ каким-то образом получены из $\alpha_{i,j}$ и также могут
содержать типовые переменные (операция \texttt{infer}). После чего, если $A$
более ничем не ограничено, либо соответствует типу входного параметра,
левая часть приравнивается к правой (операция \texttt{guess}):

\begin{equation*}
A^{\wedge} = cons(\beta_1, \dotsc, \beta_m)
\end{equation*}

Однако, инфинум из правой части \eqref{bigmeet} не всегда может быть вычислен
из-за присутствия среди $\alpha{i,j}$ типовых переменных.

\section{Текущий подход. Операция \texttt{explode}}

Если дана формула $A \subtype cons(\beta_1, \dotsc, \beta_m)$, то в
некоторых случаях можно воспользоваться наблюдением, что конструкторы в левой и
правой части должны совпадать. В частности, можно утверждать, что подтип списка
тоже является списком. Это позволяет сделать замену

\begin{align*}
A &= cons(B_1, \dotsc, B_m) \\
cons(B_1, \dotsc, B_m) &\subtype cons(\beta_1, \dotsc, \beta_m) \text{, где}
\end{align*}

$B_i$ -- новые типовые переменные. Далее раскрывается конструктор $cons$
(операция \texttt{expand}), с учётом монотонной или антимонотонной зависимости
$cons$ от аргументов (в последнем случае отношение $\subtype$ будет
перевёрнуто):

\begin{equation*}
B_i \subtype \beta_i
\end{equation*}

Возвращаясь к \eqref{upper_bounds}, получаем:

\begin{equation*}
\forall ~ j ~ B_i \subtype \alpha_{i,j}
\end{equation*}

Недостатком такого подхода является то, что он применим только для некоторых
типовых конструкторов, имеющих фиксированное количество параметров, а именно:
множеств, списков и ассоциативных массивов, т.к., в ином случае, замену для $A$
подобрать невозможно.

\section{Предлагаемый подход. Типы экстремумов}

Допустим, существовала бы возможность вычислить экстремумы типовых переменных
во всех случаях. Это позволило бы отказаться от операции \texttt{explode}, т.к.
вся необходимая функциональность реализовалась бы благодаря \texttt{infer} и
\texttt{guess}, причём была бы обеспечена поддержка всех типовых конструкторов,
таких как структуры, объединения, предикаты и т.п.

Единственное, что мешает вычислению экстремумов для любого набора типов, --
наличие в нём типовых переменных. Инфинум и, соответственно, супремум для них
определяются следующим образом:

\begin{align*}
A \meet \bot &= \bot&       A \join \bot &= A \\
A \meet \top &= A&          A \join \top &= \top \\
A \meet A &= A&             A \join A &= A
\end{align*}

Для произвольного же аргумента $\beta$ значения выражений $A \meet \beta$ и
$A \join \beta$ не определены. Доопределим их, введя новые типовые
конструкторы: \textit{meet} и \textit{join}. Задаваемые этими конструкторами
типы не зависят от порядка аргументов (в следствие коммутативности операций
$\meet$ и $\join$).

\begin{equation}
\label{meet}
\begin{split}
A \meet \beta &= meet(A, \beta) \\
meet(\alpha) &= \alpha \\
meet(\alpha_1, \dotsc, \alpha_n) \meet meet(\beta_1, \dotsc, \beta_m) &=
    (meet(\alpha_1, \dotsc, \alpha_n) \meet \beta_m) \meet
        meet(\beta_1, \dotsc, \beta_{m - 1}) \\
\beta \meet meet(\alpha_1, \dotsc, \alpha_n) =
meet(\alpha_1, \dotsc, \alpha_n) \meet \beta &=
    \begin{cases}
        \bot
            & \text{($\exists ~ i ~ \alpha_i \meet \beta = \bot$)} \\
        meet(\alpha_1, \dotsc, \alpha_i \meet \beta, \dotsc, \alpha_n)
            & \text{($\alpha_i \meet \beta$ -- не \textit{meet}-тип)} \\
        meet(\alpha_1, \dotsc, \alpha_n, \beta)
            & \text{($\forall ~ i ~ \alpha_i \meet \beta$ -- \textit{meet}-тип)}
    \end{cases} \\
\beta \join meet(\alpha_1, \dotsc, \alpha_n) =
meet(\alpha_1, \dotsc, \alpha_n) \join \beta &=
    \begin{cases}
        \beta & \text{($\exists ~ i ~ \alpha_i \subtype \beta$)} \\
        meet(\alpha_1, \dotsc, \alpha_n)
            & \text{($\forall ~ i ~ \beta \subtype \alpha_i$)} \\
        join(meet(\alpha_1, \dotsc, \alpha_n), \beta) & \text{(иначе)}
    \end{cases} \\
meet(\alpha_1, \dotsc, \alpha_n) \subtype \beta &\Leftarrow
    \exists ~ i ~ \alpha_i \subtype \beta \\
\beta \subtype meet(\alpha_1, \dotsc, \alpha_n) &\Leftarrow
    \forall ~ i ~ \beta \subtype \alpha_i \\
meet(\alpha_1, \dotsc, \alpha_n) \subtype meet(\beta_1, \dotsc, \beta_m) &\Leftarrow
    \forall ~ j ~ \exists ~ i ~ \alpha_i \subtype \beta_j
\end{split}
\end{equation}

\begin{equation}
\label{join}
\begin{split}
A \join \beta &= join(A, \beta) \\
join(\beta) &= \beta \\
join(\alpha_1, \dotsc, \alpha_n) \join join(\beta_1, \dotsc, \beta_m) &=
    (join(\alpha_1, \dotsc, \alpha_n) \join \beta_m) \join
        join(\beta_1, \dotsc, \beta_{m - 1}) \\
\beta \join join(\alpha_1, \dotsc, \alpha_n) =
join(\alpha_1, \dotsc, \alpha_n) \join \beta &=
    \begin{cases}
        \top
            & \text{($\exists ~ i ~ \alpha_i \join \beta = \top$)} \\
        join(\alpha_1, \dotsc, \alpha_i \join \beta, \dotsc, \alpha_n)
            & \text{($\alpha_i \join \beta$ -- не \textit{join}-тип)} \\
        join(\alpha_1, \dotsc, \alpha_n, \beta)
            & \text{($\forall ~ i ~ \alpha_i \join \beta$ -- \textit{join}-тип)}
    \end{cases} \\
\beta \meet join(\alpha_1, \dotsc, \alpha_n) =
join(\alpha_1, \dotsc, \alpha_n) \meet \beta &=
    \begin{cases}
        \beta & \text{($\exists ~ i ~ \beta \subtype \alpha_i$)} \\
        join(\alpha_1, \dotsc, \alpha_n)
            & \text{($\forall ~ i ~ \alpha_i \subtype \beta$)} \\
        meet(join(\alpha_1, \dotsc, \alpha_n), \beta) & \text{(иначе)}
    \end{cases} \\
join(\alpha_1, \dotsc, \alpha_n) \subtype \beta &\Leftarrow
    \forall ~ i ~ \alpha_i \subtype \beta \\
\beta \subtype join(\alpha_1, \dotsc, \alpha_n) &\Leftarrow
    \exists ~ i ~ \beta \subtype \alpha_i \\
join(\alpha_1, \dotsc, \alpha_n) \subtype join(\beta_1, \dotsc, \beta_m) &\Leftarrow
    \forall ~ i ~ \exists ~ j ~ \alpha_i \subtype \beta_j
\end{split}
\end{equation}

% TODO Неплохо бы доказать, что это не противоречит аксиомам инфинума/супремума
% для решётки. В особенности, волнует соблюдение ассоциативности.

Для примера, допустим, что некоторая типовая переменная ограничена парой типов,
хотя бы один из которых зависит от другой типовой переменной:

\begin{align*}
A^{\wedge} &\subtype struct(int ~ a, int ~ b) \\
A^{\wedge} &\subtype struct(B ~ a, nat ~ b)
\end{align*}

В операции \texttt{infer} мы попробуем вычислить инфинум структур из правых
частей. Это удастся сделать благодаря использованию типа \textit{meet}:

\begin{align*}
A^{\wedge} &\subtype struct(int ~ a, int ~ b) \meet struct(B ~ a, nat ~ b) \\
A^{\wedge} &\subtype struct(int \meet B ~ a, int \meet nat ~ b) \\
A^{\wedge} &\subtype struct(meet(B, int) ~ a, nat ~ b)
\end{align*}

После чего в \texttt{guess} вхождения $A$ будут заменены на $struct(meet(B,
int) ~ a, nat ~ b)$. Если тип $B$ также будет перезаписан, $meet(B, int)$
необходимо заменить на вычисленное значение инфинума. Добавим следующие правила
для перезаписи:

\begin{align*}
meet(\alpha_1, \dotsc, \alpha_i(A), \dotsc, \alpha_n) \Bigl\lvert_{A = \beta} &=
    meet(\alpha_1, \dotsc, \alpha_{i - 1}, \alpha_{i + 1}, \dotsc, \alpha_n) \meet (
    \alpha_i(A) \Bigl\lvert_{A = \beta}) \\
join(\alpha_1, \dotsc, \alpha_i(A), \dotsc, \alpha_n) \Bigl\lvert_{A = \beta} &=
    join(\alpha_1, \dotsc, \alpha_{i - 1}, \alpha_{i + 1}, \dotsc, \alpha_n) \join (
    \alpha_i(A) \Bigl\lvert_{A = \beta})
\end{align*}

Это позволит, при известной подстановке для $B$ завершить вывод:

\begin{align*}
A^{\wedge} &= struct(meet(B, int) ~ a, nat ~ b) \\
B &= real \\[1ex]
&\text{превращается в} \\[1ex]
A^{\wedge} &= struct(int ~ a, nat ~ b) \\
B &= real
\end{align*}

\section{Замечания относительно планируемой реализации}

Нет необходимости заводить новую операцию для раскрытия \textit{meet}- и
\textit{join}-типов, потому что они автоматически будут перезаписаны в
\texttt{rewriteType()}, как только всё необходимое для этого окажется вычислено.

\vspace{1em}

Левой частью формул \eqref{upper_bounds} не обязана быть типовая переменная, на
вычисление верхней грани это никак не влияет. Однако, если бы вместо
переменной там оказался конструктор, формулы бы были обработаны ещё до
\texttt{infer}, в операциях \texttt{eval} либо \texttt{expand}.

\vspace{1em}

Как видно из \eqref{meet} и \eqref{join}, вычислить отношение $\subtype$, если
в формуле участвуют \textit{meet}- и \textit{join}-типы, можно не всегда.

\vspace{1em}

Не кажется целесообразным создавать \textit{meet}- и \textit{join}-типы в узлах
при построении графового представления решётки типов. Следовательно, требуется
передавать флаг во все \texttt{getMeet()} и \texttt{getJoin()}, контролирующий
поведение при вычислении экстремумов с участием типовых переменных. Флаг может
также в последствие быть перенесён в специальный контекст, разделяемый между
вызовами \texttt{getMeet()} и \texttt{getJoin()}.

\vspace{1em}

Возможно, имеет смысл включать вычисление \textit{meet}- и \textit{join}-типов
только в операции \texttt{guess}, и, если удалось найти подходящий экстремум,
хотя бы и содержащий такие типы, сразу выполнять замену.

\end{document}
